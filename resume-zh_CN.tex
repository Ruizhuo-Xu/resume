% !TEX TS-program = xelatex
% !TEX encoding = UTF-8 Unicode
% !Mode:: "TeX:UTF-8"

\documentclass{resume}
\usepackage{zh_CN-Adobefonts_external} % Simplified Chinese Support using external fonts (./fonts/zh_CN-Adobe/)
% \usepackage{NotoSansSC_external}
% \usepackage{NotoSerifCJKsc_external}
% \usepackage{zh_CN-Adobefonts_internal} % Simplified Chinese Support using system fonts
\usepackage{linespacing_fix} % disable extra space before next section
\usepackage{cite}
\usepackage{color}
\definecolor{lightgray}{gray}{0.9}
\definecolor{linkcolor}{RGB}{0, 100, 200} % 自定义链接颜色

\begin{document}
\pagenumbering{gobble} % suppress displaying page number

\name{徐睿卓}

\basicInfo{
  \email{ruizhuoxu@163.com} \textperiodcentered\ 
  \phone{(+86) 13250921863} \textperiodcentered\ 
  \faBirthdayCake{}{2000.08.24} \textperiodcentered\
  \faHome{}{浙江湖州}
  }
 
\section{\faGraduationCap\  教育背景}
\datedsubsection{\textbf{北京邮电大学}, 人工智能学院,模式识别实验室}{2022.09 -- 至今}
\textit{在读硕士研究生}, \textit{2025.06毕业},\
\textit{导师:邓伟洪教授\href{http://whdeng.cn/}{\textcolor{linkcolor}{\faExternalLink{}}}(引用量1.2万+)}

\colorbox{lightgray}{\textit{一等学业奖学金}, \textit{院级优秀研究生},
\textit{1篇SCI一区论文第一作者}}

\datedsubsection{\textbf{浙江工业大学}, 信息工程学院,通信工程}{2018.09 -- 2022.06}
\textit{学士}, \colorbox{lightgray}{\textit{绩点排名:1 / 115,
获得推荐免试攻读研究生资格, 省级优秀毕业生}}

\colorbox{lightgray}{\textit{省政府奖学金}, \textit{校级优秀学生},
\textit{全国大学生智能汽车竞赛-双车接力组(全国一等奖)}}

\section{\faUsers\ 实习/项目经历}
\datedsubsection{\textbf{奇虎360\ -\ 北京\ -\ 深度学习应用部\ -\ 算法实习生}}{2021年12月 -- 2022年05月}
\colorbox{lightgray}{研究方向:\ 基于对比学习的语音模态和文本模态对齐}
\href{https://arxiv.org/abs/2204.10461}{\textcolor{linkcolor}{\faExternalLink}}
\begin{itemize}[parsep=1.0ex]
  \item 提出了一个新颖的端到端模型,将语音模态和文本模态相结合用于口语语言理解(SLU)任务;
  \item 利用CIF机制,实现了语音特征序列和文本特征序列长度的对齐;
        利用InfoNCE损失,通过将语音特征序列和文本特征序列对应位置的token作为正样本对,
        其它位置作为负样本对,进行对比学习,实现了两者特征空间的对齐,
        在保留语音特有信息的同时充分发挥了语言模型的强大能力;
  \item 相较于其他两阶段的方法或只使用语音模型的方法,本方法在口语语言理解数据集的
        情感分析子集上取得了1.15\%的召回率和0.82\%的F1分数提升;
\end{itemize}
\datedsubsection{\textbf{中国移动研究院\ -\ 北京\ -\ 合作项目}}{2022年09月 -- 2023年12月}
\colorbox{lightgray}{研究方向:\ 基于深度相机的人脸和人体分析}
\href{https://www.sciencedirect.com/science/article/pii/S0031320323006349}{\textcolor{linkcolor}{\faExternalLink{}}}
\begin{itemize}[parsep=1.0ex]
  \item 针对消费级深度相机采集得到的低质量人脸深度图像,实现了一个完整的数据预处理和数据增强
        管道,有效地降低了数据噪声的干扰并扩增了训练数据量,改善了基于深度图像的人脸识别性能;
  \item 利用隐式神经表示技术,提出了一个新颖的深度人脸图像去噪网络,
        将空间坐标信息作为深度人脸去噪和细化的先验,有效地改善了深度人脸图像的质量
        并提高了人脸识别准确率;
  \item 提出了一个轻量级的分组卷积融合模块,实现了深度图模态和法线图模态在特征层面的有效融合,
        有利于模型对人脸形状和姿态的感知,进一步提高了人脸识别准确率。
  \item 相较于之前最先进的工作,本项目所提方法在深度人脸图像的去噪和细化指标PSNR、
        SSIM以及RMSE上分别提升了0.56db,0.96和0.117,总体人脸识别率提升了4.27\%。
\end{itemize}

\section{\faInfo\ 论文}
% increase linespacing [parsep=0.5ex]
\begin{itemize}[parsep=0.5ex]
  \item \textbf{Depth Map Denoising Network and Lightweight Fusion Network for Enhanced 3D Face Recognition.}
        \textcolor{magenta}{(Pattern Recognition 2024 - 第1作者 -> 模式识别顶级期刊)}
        \href{https://www.sciencedirect.com/science/article/pii/S0031320323006349}{\textcolor{linkcolor}{\faExternalLink{}}}

        \textit{\textbf{Ruizhuo Xu}, Ke Wang, Chao Deng, Mei Wang Junlan Feng, Weihong Deng et al.}

        \colorbox{lightgray}{三维人脸识别、深度图去噪、隐式神经表示}

        \begin{itemize}[parsep=0.5ex]
          \item 首次将隐式神经表示技术引入深度人脸图像去噪领域,利用空间坐标信息指导深度人脸去噪;
                引入位置编码,并提出了一个多尺度解码融合策略,有效地提升了深度人脸去噪的性能;
          \item 提出了一个轻量级三维人脸识别网络,通过一个多分支卷积融合模块实现了深度图模态和法线图
                模态的深度融合,在提高三维人脸表征质量的同时,平衡了计算开销;
          \item 利用提出的深度人脸去噪网络和人脸识别网络,在多个三维人脸数据集上均取得了SOTA的去噪
                和识别性能;
        \end{itemize}
        
  \item \textbf{Skeleton2vec: A Self-supervised Learning Framework with Contextualized Target Representations for Skeleton Sequence}
        \textcolor{magenta}{(CVPR 2024在投 - 第1作者)}
        \href{https://arxiv.org/abs/2401.00921}{\textcolor{linkcolor}{\faExternalLink{}}}

        \textit{\textbf{Ruizhuo Xu}, Linzhi Huang, Mei Wang, Jiani Hu, Weihong Deng}

        \colorbox{lightgray}{自监督预训练、掩码预测、基于骨架的行为识别}

        \begin{itemize}[parsep=0.5ex]
          \item 之前基于掩码预测的骨架序列预训练工作往往采用局部的、低层次的预测目标(如:原始关节点),这是次优的;
                为此,我们提出了Skeleton2vec框架,利用EMA更新的教师编码器生成全局上下文化的高层次
                特征表示作为预测目标,迫使编码器学习到的表征具有更强的时空联系。
          \item 针对骨架序列具有较高时空关联性可能会产生的信息泄露问题,我们提出了基于运动感知的管道遮蔽策略
                (Motion-aware Tube Masking),迫使模型建模更好的长程时空联系,并持续关注运动语义丰富的区域;
          \item 在三个大规模3D骨架行为识别数据集上的多个测试协议下,均取得了SOTA的性能;
        \end{itemize}
        
  \item \textbf{WaBERT: A Low-resource End-to-end Model for Spoken Language Understanding and Speech-to-BERT Alignment}
        \textcolor{magenta}{(共同1作)}
        \href{https://arxiv.org/abs/2204.10461}{\textcolor{linkcolor}{\faExternalLink{}}}

        \textit{Lin Yao, Jianfei Song, \textbf{Ruizhuo Xu}, Yingfang Yang, Zijian Chen, Yafeng Deng}

        \colorbox{lightgray}{模态对齐、对比学习、口语语言理解}

        \begin{itemize}[parsep=0.5ex]
          \item 口语语言理解任务主要有两种主流方法:(1) 两阶段法:首先将语音通过语音识别模型转成文本,作为语言模型的输入,
                微调语言模型做下游任务;(2) 端到端法:直接微调预训练好的语音模型做下游任务;前者会丢失语音特有的信息
                并易受语音识别错误的影响;后者缺乏语言模型强大的语言理解能力;为此,我们提出一个新的端到端方案,
                结合语音模型和语言模型用于口语语言理解任务;在不丢失语音特有信息的同时,充分发挥语言模型的能力;
          \item 利用CIF机制实现语音模态和文本模态特征序列长度的对齐,利用InfoNCE Loss对齐语音模态和文本模态的
                特征空间; 推理时,语音输入经过语音编码器提取声学特征,提取得到的特征再作为语言编码器的输入,
                输出下游任务的结果;
        \end{itemize}
        
\end{itemize}

\section{\faCogs\ 专业技能}
% increase linespacing [parsep=0.5ex]
\begin{itemize}[parsep=0.5ex]
  \item 编程能力: 了解Python、C/C++、Linux、Latex、Git、Vim;
  \item 深度学习: 熟悉使用Pytorch开发深度学习模型,在算法改进和工程开发方面有相关实操经验;
  \item 个人证书: 英语(CET-4、CET-6);
\end{itemize}

\section{\faHeartO\ 主要荣誉}
\begin{itemize}[parsep=0.5ex]
  \item \datedline{第十六届全国大学生智能汽车竞赛,\textit{全国一等奖(4 / 225)}}{2021.08}
  \item \datedline{第十五届全国大学生智能汽车竞赛,\textit{浙江省三等奖}}{2020.08}
  \item \datedline{省级优秀毕业生}{2022.06}
  \item \datedline{院级优秀研究生}{2023.09}
  \item \datedline{省政府奖学金 \& 校级优秀学生}{2019.09 / 2020.09 / 2021.09}
\end{itemize}


%% Reference
%\newpage
%\bibliographystyle{IEEETran}
%\bibliography{mycite}
\end{document}
