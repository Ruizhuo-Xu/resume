% !TEX TS-program = xelatex
% !TEX encoding = UTF-8 Unicode
% !Mode:: "TeX:UTF-8"

\documentclass{resume}
\usepackage{zh_CN-Adobefonts_external} % Simplified Chinese Support using external fonts (./fonts/zh_CN-Adobe/)
% \usepackage{NotoSansSC_external}
% \usepackage{NotoSerifCJKsc_external}
% \usepackage{zh_CN-Adobefonts_internal} % Simplified Chinese Support using system fonts
\usepackage{linespacing_fix} % disable extra space before next section
\usepackage{cite}
\usepackage{color}
\definecolor{lightgray}{gray}{0.9}

\begin{document}
\pagenumbering{gobble} % suppress displaying page number

\name{徐睿卓}

\basicInfo{
  \email{ruizhuoxu@163.com} \textperiodcentered\ 
  \phone{(+86) 13250921863} \textperiodcentered\ 
  \faBirthdayCake{}{2000.08.24} \textperiodcentered\
  \faHome{}{浙江湖州}
  }
 
\section{\faGraduationCap\  教育背景}
\datedsubsection{\textbf{北京邮电大学}, 人工智能学院,模式识别实验室}{2022.09 -- 至今}
\textit{在读硕士研究生}, \textit{2025.06毕业},\
\textit{导师:邓伟洪教授\href{http://whdeng.cn/}{\faExternalLink{}}(引用量1.2万+)}

\colorbox{lightgray}{\textit{一等学业奖学金}, \textit{院级优秀研究生},
\textit{1篇SCI一区论文第一作者}}

\datedsubsection{\textbf{浙江工业大学}, 信息工程学院,通信工程}{2018.09 -- 2022.06}
\textit{学士}, \colorbox{lightgray}{\textit{绩点排名:1/115,
获得研究生推免资格, 省级优秀毕业生}}

\colorbox{lightgray}{\textit{省政府奖学金}, \textit{校级优秀学生}
\textit{全国大学生智能汽车竞赛-双车接力组(全国一等奖)}}

\section{\faUsers\ 实习/项目经历}
\datedsubsection{\textbf{奇虎360\ -\ 北京\ -\ 深度学习应用部\ -\ 算法实习生}}{2021年12月 -- 2022年05月}
\colorbox{lightgray}{研究方向:\ 基于对比学习的语音模态和文本模态对齐}
\begin{itemize}[parsep=1.0ex]
  \item 提出了一个新颖的端到端模型,将语音模态和文本模态相结合用于口语语言理解(SLU)任务;
  \item 利用CIF机制,实现了语音特征序列和文本特征序列长度的对齐;
        利用InfoNCE损失,通过将语音特征序列和文本特征序列对应位置的token作为正样本对,
        其它位置作为负样本对,进行对比学习,实现了两者特征空间的对齐,
        在保留语音特有信息的同时充分发挥了语言模型的强大能力;
  \item 相较于其他两阶段的方法或只使用语音模型的方法,本方法在口语语言理解数据集的
        情感分析子集上取得了1.15\%的召回率和0.82\%的F1分数提升;
\end{itemize}
% Reference Test
%\datedsubsection{\textbf{Paper Title\cite{zaharia2012resilient}}}{May. 2015}
%An xxx optimized for xxx\cite{verma2015large}
%\begin{itemize}
%  \item main contribution
%\end{itemize}

\section{\faCogs\ IT 技能}
% increase linespacing [parsep=0.5ex]
\begin{itemize}[parsep=0.5ex]
  \item 编程语言: C == Python > C++ > Java
  \item 平台: Linux
  \item 开发: xxx
\end{itemize}

\section{\faHeartO\ 获奖情况}
\datedline{\textit{第一名}, xxx 比赛}{2013 年6 月}
\datedline{其他奖项}{2015}

\section{\faInfo\ 其他}
% increase linespacing [parsep=0.5ex]
\begin{itemize}[parsep=0.5ex]
  \item 技术博客: http://blog.yours.me
  \item GitHub: https://github.com/username
  \item 语言: 英语 - 熟练(TOEFL xxx)
\end{itemize}

%% Reference
%\newpage
%\bibliographystyle{IEEETran}
%\bibliography{mycite}
\end{document}
